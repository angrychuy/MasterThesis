% chapter1.tex
% Capitulo 1. Introduccion
%==========================================================================
\chapter{Introducci\'on}

El proyecto que se documenta eval\'ua la plataforma USRP como herramienta
\'util para la ense\~nanza de las t\'ecnicas de modulaci\'on digital. La
evaluaci\'on consiste en introducir alguna secuencia de datos al sistema
(aleatoria, datos extra\'idos de un arch\'ivo, etc.), generar la modulaci\'on y
demodulaci\'on para la transmisi\'on y observar los resultados.

Los sistemas de de comunicaciones digitales implementados en su  mayor parte por
software se les conocen como Radios Definidos por
Software(SDR) \cite{mitola}. El sistema se desarroll\'o dentro de la plataforma
Linux, utilizando el software \emph{GNURadio}.

% 1.1. Antecedentes
%==========================================================================
\section{Antedecentes}%Seccion 1.1 Antecedentes

Gracias a los avances tecnologicos recientes, han surgido varios esfuerzos por
desarrollar plataformas vers\'atiles para la implementaci\'on de SDRs.
La complejidad de los algoritmos de modulaci\'on, codificaci\'on, filtrado,
etc., los hacen buenos candidatos para ser implementados en DSPs como el C6711 de la Texas
Instruments \cite{abendroth}.

Existen varias soluciones comerciales que ofrecen un nivel m\'as avanzado y
completo para la investigaci\'on y desarrollo de aplicaciones de comunicaciones
basadas en SDR. Pentek ofrece plataformas basadas en la interfaz PCI y PCIe
entre otras \cite{pentek}. Estas plataformas utilizan FPGAs de Xilinx para 
implementar los bloques de DDC y DUC, as� como tamb\'en filtros de interpolaci�n. La compa�\'ia
ofrece drivers para Windows y Linux, ofreciendo as\'i mucha flexibilidad de
escoger un ambiente adecuado para el desarrollo de la aplicaci\'on.

El mundo de \emph{Open Source} tambi�n ha realizado avances notorios en este
ramo de investigaci\'on. El proyecto m\'as sobresaliente es
\emph{GNURadio} \cite{radio}. Este proyecto consiste en una librer\'ia extensa
con varias funciones matem\'aticas, as\'i como tambi\'en funciones para la
implementaci\'on de algoritmos de procesamiento digital de se�ales, basada en los lenguajes Python y C++. 
La librer\'ia motiv\'o el desarrollo de un dispositivo de bajo costo, conocido
como el USRP por sus siglas en ingles \emph{Universal Software Radio Peripheral}, con el prop\'osito de
utilizarlo para la transmisi\'on y recepci\'on de varios tipos de se�ales
generados por \emph{GNURadio} \cite{ettus}. Este dispositivo esta tambi\'en
basado en un FPGA y un par de ADCs y DACs respectivamente. La se�al es enviada a la PC por 
medio del puerto USB utilizando el controlador CY7C68013 de la compa�\'ia
Cypress. A pesar de ser un sistema \emph{Open Source}, el USRP y \emph{GNURadio}
han sido utilizados en diversos campos de investigaci\'on y desarrollo como
acad\'emicos, comerciales, militares y gubernamentales.

% 1.2. Planteamiento del problema
%==========================================================================
\section{Planteamiento del problema}
Los conceptos del \'area de comunicaciones digitales utilizan 
teor\'ia matem\'atica que a veces es dif\'icil de visualizar sin una 
forma de poder utilizar estos conceptos de una manera pr\'actica.
Los ambientes de desarrollo, 
como Matlab, proporcionan la habilidad de poder realizar estudios y prototipos 
de estos conceptos y sistemas utilizando modelos matem\'aticos que aproximan el 
comportamiento real de un sistema de comunicaciones. F\'isicamente no es
posible ver los efectos de la teor\'ia sin una implementaci�n real del sistema y
el equipo que se utiliza para lograrlo (analizadores de espectro, antenas, 
analizadores de sistemas de comunicaciones, DSPs, etc.) son relativamente 
caros lo cual hace dif\'icil una implementaci\'on real.
 

% 1.3. Justificaci�n
%==========================================================================
\section{Justificaci\'on}

El estudio de los conceptos de comunicaciones normalmente ha sido dado de una
manera te\'orica. En la mayor\'ia de las instituciones se realizan simulaciones
en Matlab o en alg\'un otro ambiente de desarrollo. Estos paquetes proporcionan
datos simulados del estudio en cuesti\'on, ya sea la transmisi\'on de alguna
se�al o el an\'alisis de alg\'un canal de transmisi\'on. El USRP ofrece la
oportunidad de llevar a cabo estos conceptos a un ambiente m\'as pr\'actico, es
decir, los datos obtenidos del estudio son de los efectos reales de una
implementaci\'on f\'isica de un sistema de comunicaciones. El ambiente de
desarrollo \emph{GNURadio} ofrece incluso un ambiente amigable y sencillo de
utilizar ya que el lenguaje de programaci\'on utilizado, Python, es muy similar
al estilo de programaci\'on de Matlab.

% 1.4. Importancia del estudio
%==========================================================================
\section{Importancia del estudio}

El estudio de una plataforma m\'as amigable y m\'as din\'amica para el
entendimiento de los conceptos de comunicaciones digitales es de gran importancia para el 
desarrollo acad\'emico del alumno que estudia dichos temas. Estos conceptos son 
normalmente estudiados dentro de ambientes de simulaci�\'on para ver sus efectos 
y comprobar de forma anal\'itica la teor\'ia. La habilidad de poder llevar estos 
an\'alisis te\'oricos a la pr\'actica ofrece la oportunidad importante de poder 
visualizar los resultados obtenidos en la simulaci\'on en una implementaci\'on 
real donde se puede observar con mayor exactitud los efectos y resultados 
que se est\'an estudiando. La plataforma USRP ofrece una versatilidad muy 
amplia para el estudio de sistemas de comunicaci\'on digital que f\'acilmente 
se pueden llevar a una implementaci\'on a mayor escala. Con esto, la importancia 
de este estudio se vuelve m\'as evidente cuando el alumno que utilice esta 
plataforma pueda llevar sus conocimientos b\'asicos a la pr\'actica e
implementar un sistema de comunicaciones completo de una manera r\'apida y
sencilla.

% 1.5. Objetivos
%==========================================================================
\section{Objetivos}

Los objetivos del presente proyecto son los siguientes:
\begin{itemize}
  \item Implementar la modulaci\'on y demodulaci\'on QPSK en el sistema USRP.
  \item Graficar la SNR del rendimiento del sistema.
  \item C\'alculo de la tasa de error de bit.
  \item C\'alculo y gr\'aficas de densidad espectral de potencia.
  \item Diagrama de ojo y de constelaci\'on del esquema de modulaci\'on.
\end{itemize}

% 1.6. Metodologia
%==========================================================================
\section{Metodolog\'ia}

La metodolog\'ia a seguir es la siguiente:
\begin{itemize}
  \item El desarrollo se llevara a cabo en el sistema operativo Linux
  utilizando el software \emph{GNURadio}.
  \item Instalaci\'on del sistema operativo y el software.
  \item Validaci\'on de la instalaci\'on. Ejecuci\'on de programas de prueba
  proporcionados por \emph{GNURadio}.
  \item Estudio de los bloques de procesamiento de \emph{GNURadio}.
  \item Experimentos con el sistema USRP.
  \item Generaci\'on de gr\'aficas para medir y visualizar los resultados de los
  experimentos.
\end{itemize}