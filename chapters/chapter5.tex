% chapter5.tex
% Capitulo 5. Conclusion
%==========================================================================
\chapter{Conclusiones}
\gnuradio es una plataforma muy flexible para el desarrollo de aplicaciones en el \'area de telecomunicaciones y gracias a que es
un proyecto abierto, es constantemente desarrollado y actualizado por una comunidad muy activa. El foro de discusi\'on que se
encuentra en la p\'agina de \gnuradio \cite{radio} fue de gran apoyo para el desarrollo de este trabajo debido a que la gente que
se encuentra participando activamente ayuda mucho a los nuevos usuarios por lo cual se recomienda que se utilice mucho este recurso para
cualquier persona que desee utilizar \gnuradio para sus aplicaciones.

La versi\'on de \gnuradio que se utilizo fue la mas reciente que se encuentra en el repositorio \emph{git} del proyecto. Este
repositorio es actualizado constantemente conforme los autores y la gente de la comunidad contribuye cambios y mejoras al c\'odigo
fuente y por lo tanto, ofrece la mayor cantidad de herramientas y bloques que la versi\'on oficial que se encuentra en la pagina
principal. La versi\'on que se esta utilizando en el momento puede ser f\'acilmente actualizada a la mas nueva simplemente
siguiendo las instrucciones que se presentan en el ap\'endice \ref{AppA}. De esta manera uno puede estar seguro que siempre
tendr\'a acceso a las mejoras mas recientes del c\'odigo y se podr\'an utilizar inmediatamente en la aplicaci\'on del usuario.

Los experimentos realizados durante el desarrollo de este trabajo fueron satisfactorios e ilustraron como utilizar los bloques de
procesamiento de \gnuradio para transmitir datos utilizando un esquema de modulaci\'on de $M$ estados, en este caso QPSK donde
$M=4$. Se mostro una aplicaci\'on que se puede utilizar como base para la implementaci\'on de los diferentes conceptos de
telecomunicaciones como son la creaci\'on de paquetes de datos, desarrollo de protocolos de comunicaci\'on, implementaci\'on de
otros esquemas de modulacion (BPSK, GMSK, etc.), filtrado, etc. Por ultimo se mostro una herramienta indispensable para la ayuda del
entendimiento de los conceptos de telecomunicaciones y tambi\'en como apoyo para el desarrollo de aplicaciones, que es
\emph{GNURadio Companion}.

En conclusi\'on, \gnuradio y el USRP son herramientas flexibles y de bajo costo que se pueden utilizar en un ambiente acad\'emico
para acelerar y mejorar la comprensi\'on de los conceptos de telecomunicaciones, ya sea a nivel licenciatura para el entendimiento
de los conceptos b\'asicos o a nivel maestr\'ia para la investigacio\'n de conceptos mas avanzados. La plataforma es capaz de
ajustarse a las diferentes necesidades de los alumnos e investigadores, gracias a la amplia gama de frecuencias que soporta y su
flexibilidad de implementar e incorporar las diferentes t\'ecnicas de modulaci\'on, codificaci\'on, filtrado, etc., en una
plataforma que permite observar resultados reales y en tiempo real. Esto es de gran importancia para su entendimiento y
comparaci\'on con la teoria que forma parte de la \'area de telecomunicaciones.

\section{Trabajo futuro}
Utilizando los conocimientos que se presentaron en este trabajo, se puede trabajar en mejorar el c\'odigo fuente con aportaciones
que implementen t\'ecnicas que aun no se encuentren integradas en \gnuradio. Un trabajo interesante es la implementaci\'on del
esquema QAM. Actualmente el c\'odigo fuente contiene un modulador sencillo de QAM pero el demodulador solamente contiene el
esqueleto de un bloque jer\'arquico sin ning\'un funcionamiento. Los autores aun no lo han implementado pero dejaron la estructura
del bloque lista para que alguien pueda desarrollarlo.

Algunas aplicaciones en las que se puede trabajar a futuro pueden ser las que se mencionan a continuaci\'on. La gama de
aplicaciones es amplia y estas son solamente algunas de ellas:
\begin{itemize}
  \item GPS
  \item Comunicaci\'on satelital
  \item Comunicaciones moviles
  \item T\'ecnicas de acceso como TDMA, FDMA y CDMA.
\end{itemize}