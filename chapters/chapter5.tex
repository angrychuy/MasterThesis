% chapter5.tex
% Capitulo 5. Conclusion
%==========================================================================
%TODO: Anexar comentario sobre la necesidad de enfocarse mas en la programacion para la enseñanza
\chapter{Conclusiones}
Los experimentos realizados durante el desarrollo de este trabajo fueron satisfactorios e ilustraron c\'omo utilizar los bloques
de procesamiento de \gnuradio para transmitir datos utilizando un esquema de modulaci\'on de $M$ estados, en este caso QPSK donde
$M=4$. Se mostr\'o una aplicaci\'on que se puede utilizar como base para la implementaci\'on de los diferentes conceptos de
telecomunicaciones como son la creaci\'on de paquetes de datos, desarrollo de protocolos de comunicaci\'on, implementaci\'on de
otros esquemas de modulaci\'on (BPSK, GMSK, etc.), filtrado, etc. Por \'ultimo se mostr\'o una herramienta indispensable para la
ayuda del entendimiento de los conceptos de telecomunicaciones y como apoyo para el desarrollo de aplicaciones, que es
\emph{GNURadio Companion}.

\gnuradio es una plataforma flexible para el desarrollo de aplicaciones en el \'area de telecomunicaciones y ya que es
un proyecto abierto, es constantemente desarrollado y actualizado por una comunidad activa. El foro de discusi\'on que se
encuentra en la p\'agina de \gnuradio \cite{radio} fue de gran apoyo para el desarrollo de este trabajo debido a que la gente que
se encuentra participando activamente ayuda mucho a los nuevos usuarios. Se recomienda que se utilice este
recurso para cualquier persona que desee utilizar \emph{GNURadio}.

%La versi\'on de \gnuradio que se utiliz\'o fue la m\'as reciente que se encuentra en el repositorio \emph{git} del proyecto. Este
%repositorio es actualizado constantemente conforme los autores y la gente de la comunidad contribuye cambios y mejoras al c\'odigo
%fuente y por lo tanto, ofrece la mayor cantidad de herramientas y bloques que la versi\'on oficial que se encuentra en la p\'agina
%principal. La versi\'on que se est\'e utilizando en el momento puede ser actualizada a la m\'as nueva siguiendo las instrucciones
%que se presentan en el ap\'endice \ref{AppA}. De esta manera puede estar seguro que siempre tendr\'a acceso a las mejoras mas
%recientes del c\'odigo y se podr\'an utilizar inmediatamente en la aplicaci\'on del usuario.

En conclusi\'on, \gnuradio y el USRP son herramientas flexibles y de bajo costo que se pueden utilizar en un ambiente acad\'emico
para acelerar y mejorar la comprensi\'on de los conceptos de telecomunicaciones, ya sea a nivel licenciatura para el entendimiento
de los conceptos b\'asicos o a nivel maestr\'ia para la investigaci\'on de conceptos m\'as avanzados. La plataforma es capaz de
ajustarse a las diferentes necesidades de los alumnos e investigadores, gracias a la amplia gama de frecuencias que soporta y su
flexibilidad de implementar e incorporar las diferentes t\'ecnicas de modulaci\'on, codificaci\'on, filtrado, etc., en una
plataforma que permite observar resultados reales y en tiempo real. Esto es de gran importancia para su entendimiento y
comparaci\'on con la teor\'ia que forma parte del \'area de telecomunicaciones.

\section{Trabajo futuro}
Utilizando los conocimientos que se presentaron en este trabajo, se puede trabajar en mejorar el c\'odigo fuente con aportaciones
que implementen t\'ecnicas que aun no se encuentren integradas en \emph{GNURadio}. Un trabajo interesante es la implementaci\'on
del esquema QAM. Actualmente el c\'odigo fuente contiene un modulador sencillo de QAM pero el demodulador solamente contiene el
esqueleto de un bloque jer\'arquico sin ning\'un funcionamiento. Los autores aun no lo han implementado pero dejaron la estructura
del bloque lista para que alguien pueda desarrollarlo.

Utilizando las tarjetas auxiliares mencionadas en la tabla \ref{tbl:cards} es posible realizar trabajos m\'as avanzados en
diferentes rangos de frecuencias. Algunas aplicaciones en las que se puede trabajar a futuro pueden ser las siguientes:
\begin{itemize}
  \item GPS
  \item Comunicaci\'on satelital
  \item Comunicaciones m\'oviles
  \item T\'ecnicas de acceso como TDMA, FDMA y CDMA.
\end{itemize}