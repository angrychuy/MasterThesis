% chapter4.tex
% Capitulo 4. Resultados
%==========================================================================
\chapter{Resultados}

En este cap\'itulo se muestran los resultados del estudio de la plataforma USRP. El experimento
consisti\'o en transmitir datos generados de una fuente aleatoria por software y de una fuente fija
que consiste en una imagen entre las tarjetas LFTX y LFRX por medio de un cable coaxial. Estos
datos son modulados con el esquema DQPSK como esta implementado en \gnuradio. Despu\'es se
observaron los resultados utilizando las herramientas proporcionadas por \gnuradio y el lenguaje Python.

%Material utilizado
%==========================================================================
\section{Material utilizado}
El material que se utiliz\'o para realizar el experimento fue el siguiente:

\begin{itemize}
  \item 1 USRP
  \item Tarjetas LFTX y LFRX
  \item Cable coaxial con terminales SMA a SMA
  \item Cable USB
  \item Computadora laptop con las siguientes caracter\'isticas:
  \begin {itemize}
    \item CPU Intel Core 2 Duo T5550 a 1.83Ghz
    \item 4GB memoria DDR2
    \item Sistema operativo Ubuntu 10.01
  \end{itemize}
\end{itemize}

Los programas \verb|benchmark_tx.py| y \verb|benchmark_rx.py| de la carpeta de ejemplos del codigo
fuente fueron utilizados para realizar la transmisi\'on, ya que estos ejemplos son programas
completos y maduros que ayudan a ilustrar varias caracter\'isticas de la programaci\'on con
\gnuradio. Los programas \verb|benckmark_tx2.py| y \verb|benchmark_rx2.py| tienen la misma
estructura que su versi\'on anterior pero estos implementan el modulador y demodulador utilizando la
t\'ecnica de filtros polif\'asicos. Esta versi\'on, incluyendo los bloques que implementan estos
filtros, fueron incluidos en el c\'odigo fuente el 23 de Marzo de 2010 y fueron incluidos en este
estudio para comparar el rendimiento de ambas versiones.


%Parametros utilizados
%==========================================================================
\section{Par\'ametros utilizados}

Para ejecutar los ejemplos \verb|benchmark_tx.py| y \verb|benchmark_rx.py| es neceario moverse al
directorio donde se encuentran. La ruta es:
\begin{center}
\verb|<dir_donde_esta_el_codigo_fuente>|/\verb|gnuradio-examples|/\verb|python|/\verb|digital|
\end{center}
Los programas se ejecutan por medio de la consola de texto utilizando la notacion \verb|./| para
indicar que se va correr un ejecutable. La sintaxis completa es:
\begin{center}
\verb|./benchmark_tx.py <opciones>|
\end{center}
donde opciones son algunas opciones que se pueden pasar al programa para controlar su
funcionamiento. Para ver todas las opciones que soporta se puede utilizar el siguiente comando:
\begin{center}
\verb|./benchmark_tx.py --help|
\end{center}

Los opciones que se utilizaron para el transmisor y receptor son los siguientes:

\section*{TRANSMISOR}
\begin{itemize}
  \item \textbf{-m}: Especifica el modulador que se va utilizar. La opci\'on que se utilizo fue
  DQPSK y DBPSK. BPSK se utilizo para realizar la comparaci\'on entre ambos esquemas.
  \item \textbf{-r}: Especifica la taza de bits. Se utilizaron varios valores desde 100kbps hasta
  10kbps. Este valor depende mucho de la PC que se est\'e utilizando ya que valores muy altos
  dependen del rendimiento del CPU.
  \item \textbf{--tx-amplitude}: Especifica la amplitud de la se\~nal generada por el DAC en el
  USRP. Sus valores van de 0 a 1. Se utilizo el valor default de 0.25.
  \item \textbf{--excess-bw}: Especifica el par\'ametro $\beta$ del filtro acoplador de coseno
  elevado. Su rango de valores es de 0 a 1. Se utilizaron tres valores para el an\'alisis: 0.35,
  0.50 y 0.75.
  \item \textbf{-f}: Especifica la frecuencia con la cual se van a sintonizar las tarjetas
  auxiliares. Para este estudio se estuvo utilizando la frecuencia m\'axima que soportan las
  tarjetas LFTX y LFRX que es 30Mhz.
  \item \textbf{-S}: Especifica la cantidad de muestras por s\'imbolo. El valor default es de 2. Se
  observo que se tuvo un mejor rendimiento con un valor de 4.
\end{itemize}

\section*{RECEPTOR}
\begin{itemize}
  \item \textbf{-m}: Especifica el demodulador que se va utilizar. Igual que en el transmisor se
  utiliz\'o DQPSK y DBPSK.
  \item \textbf{--gain-mu}: Especifica la ganancia que se utiliza en el lazo de costas para la
  detecci\'on de la fase. Se utilizo un valor de 0.5.
  \item \textbf{-f}: Especifica la frecuencia a la que se sintoniza la tarjeta RX. Se utiliz\'o un
  valor de 30Mhz.
  \item \textbf{-S}: Especifica la cantidad de muestras por s\'imbolo utilizada en el demodulador.
  Igual que en el transmisor se utilizo un valor de 4.
  \item \textbf{-r}: Especifica la taza de bits. Se utilizaron los mismos valores que en el
  transmisor.
  \item \textbf{--excess-bw}: Especifica el par\'ametro $\beta$ del filtro acoplador de coseno
  elevado. Se utilizaron los mismos valores que en el transmisor.
\end{itemize}

Cabe mencionar que debido a que los dos programas son independientes uno del otro, es necesario
igualar algunos par\'ametros para lograr una transmisi\'on exitosa. Los par\'ametros \verb|--f|,
\verb|-S|, \verb|-r| y \verb|--excess-bw| son un ejemplo de estos parametr\'os. Todos los
par\'ametros tienen valores default si no se desea modificarlos pero uno de ellos es obligatorio
especificarlo y es el de la frecuencia de sintonizaci\'on \verb|-f|. Si se ejecutan los programas
sin especificar este par\'ametro, aunque se especifiquen los otros, marcara un error pidiendo que
se especifique la frecuencia de operacion. 
%Taza de error de bits
%==========================================================================
\section{Taza de error de bits}

%TODO: Realizar diagrama de los grafos que se desarrollaron e incluir las graficas del ber.
El an\'alisis de la tasa de error de bits (Bit error rate o BER por sus siglas en ingles) se realizo
para ambos modelos de DQPSK proporcionados por \emph{GNURadio}. El programa que se utilizo fue
basado en un estudio generado y publicado en la pagina de \gnuradio sobre una versi\'on mas optima
del modulador GMSK. El programa fue modificado para que pudiera analizar y comparar ambos esquemas
DBPSK y DQPSK en sus dos versiones: costas/MM y filtros polifasicos.

La estructura del programa est\'a dividida en 3 etapas. La primera etapa genera una se\~nal de una
fuente de n\'umeros pseudo-aleatorios y luego la modula utilizando el modulador que se va a evaluar.
Los datos arrojados del modulador son enviados a archivos que se utilizaran como la entrada para las
siguientes etapas. El programa define una clase llamada \verb|SigGen| derivada de la clase
\verb|top_block| para definir un grafo de flujo de datos. La estructura del grafo se muestra en la
figura \ref{fig:siggen}.

\begin{figure}[htp]
  \centering
  \vspace{0.3in}
  \begin{tikzpicture}[scale=0.8, transform shape, node distance=0.5cm and 0.4cm]
  	\node (glfsr) [grblock] {\footnotesize{Fuente de n\'umeros pseudo-aleatorios}};
  	\node (limiter) [grblock, right=of glfsr] {\footnotesize{Limitador de datos}}
  	edge [<-] (glfsr);
  	\node (srcbits) [grblock, below=of limiter] {\footnotesize{src\_bits.dat}}
  	edge [<-] (limiter);
  	\node (unpacktopack) [grblock, right=of limiter] {\footnotesize{Empaquetado de bits}}
  	edge [<-] (limiter);
  	\node (newmod) [grblock, above right=of unpacktopack] {\footnotesize{Mod\_Demod \\ Poly}};
  	\draw [->] (unpacktopack.east) -- ++(0.2,0) |- (newmod);
  	\node (cleanbb) [grblock, right=of newmod] {\footnotesize{limpio\_bb.dat}}
  	edge [<-] (newmod);
  	\node (oldmod) [grblock, below right=of unpacktopack] {\footnotesize{Mod\_Demod \\ Costas\_MM}};
  	\draw [->] (unpacktopack.east) -- ++(0.2,0) |- (oldmod);
  	\node (cleanoldbb) [grblock, right=of oldmod] {\footnotesize{limpio\_oldbb.dat}}
  	edge [<-] (oldmod);
  \end{tikzpicture}
  \vspace{0.3in}
  \caption{Grafo generador de datos modulados para la evaluaci\'on del BER.}
  \label{fig:siggen}
\end{figure}

Los bloques se describen de la siguiente manera:

\begin{itemize}
  \item \textbf{Fuente de n\'umeros pseudo-aleatorios}: El bloque se genera por medio de la clase
  \verb|gr.glfsr_source_b|. Esta clase implementa un registro de desplazamiento con
  retroalimentaci\'on lineal en modo Galois. El sufijo ``b'' indica que el generador entrega valores
  binarios. El par\'ametro que se le especifica es el numero de bits de precisi\'on para generar
  una secuencia de bits de longitud $2^{N-1}$, el cual expresa la m\'axima cantidad de valores
  que puede generar antes de ciclarse \cite{xilinx}. 
  \item \textbf{Limitador de datos}: Este bloque se genera por medio de la clase \verb|gr_head|, el
  cual su funci\'on es tomar una cierta cantidad de valores en su entrada y descartar todo lo
  dem\'as. A esta secuencia de valores se le anexa el valor especial \emph{EOF} que especifica el
  fin del flujo dentro del grafo. Esto causar\'a que el grafo termine su ejecuci\'on una vez que
  este valor se propague a todos los bloques. Sus par\'ametros el tipo de datos con el que va
  trabajar y la cantidad de muestras que va dejar pasar de su entrada a su salida.
  \item \textbf{Source\_bits.dat, Limpio\_bb.dat, Limpio\_oldbb.dat}: Estos bloques utilizan la
  clase \verb|gr.file_sink| para enviar el flujo de datos a un archivo. En este grafo se generan tres
  archivos para guardar los bits originales y los bits modulados con los dos tipos de moduladores
  de \emph{GNURadio}. Sus par\'ametros son el tipo de datos que van a escribir y el nombre del
  archivo.
  \item \textbf{Empaquetado de bits}: Este bloque se implementa utilizando la clase \\
  \verb|gr.unpacked_to_packed_bb|. Se encarga en juntar los bits que entrar y formar bytes completos
  en la salida. Estos bytes representan la informaci\'on original.
  \item \textbf{Mod\_Demod Poly y Mod\_Demod Costas\_MM}: Estos bloques representan los moduladores
  que se evaluaron durante el experimento y se implementan con las clases \verb|blks2.dqpsk_mod| y
  \verb|blks2.dqpsk2_mod| respectivamente.
\end{itemize}

El c\'odigo que implementa este grafo se muestra en el listado \ref{ex:siggen}.

\begin{lstlisting}[float, label=ex:siggen, caption={C\'odigo que implementa el grafo
generador de se\~nales.}, breaklines=true]
class SigGen(gr.top_block):
  def __init__(self,num_bits=5000,pn_degree=23,xmit_bt=0.25,samp_per_symbol=8):
    gr.top_block.__init__(self, 'SigGen')
	
	src = gr.glfsr_source_b(pn_degree)
	src_limiter = gr.head(gr.sizeof_char, num_bits)

    bit_packer = gr.unpacked_to_packed_bb(1, gr.GR_MSB_FIRST)
    mod = blks2.dqpsk2_mod(samp_per_symbol, xmit_bt)

    bit_packer_old = gr.unpacked_to_packed_bb(1, gr.GR_MSB_FIRST)
    mod_old = blks2.dqpsk_mod(samp_per_symbol, xmit_bt)

    self.connect(src, src_limiter)
    self.connect(src_limiter, gr.file_sink(gr.sizeof_char, 'src_bits.dat'))
    self.connect(src_limiter, bit_packer, mod, gr.file_sink(gr.sizeof_gr_complex, 'limpio_bb.dat'))
    self.connect(src_limiter, bit_packer_old, mod_old, gr.file_sink(gr.sizeof_gr_complex, 'limpio_old_bb.dat'))
\end{lstlisting}

La segunda etapa toma la se\~nal modulada de los archivos y los pasa a trav\'es de un grafo que
aplica ruido Gaussiano para simular un canal de transmisi\'on AWGN. El grafo se muestra en la figura
\ref{fig:noisegen}.

\begin{figure}[htp]
  \centering
  \vspace{0.3in}
  \begin{tikzpicture}[scale=0.8, transform shape, node distance=15mm and 20mm]
  	\node (sigsource) [grblock] {\footnotesize{Fuente de datos de un archivo}};
  	\node (noisesource) [grblock, below=of sigsource] {\footnotesize{Fuente de ruido}};
  	\node (sum) [op, below right=of sigsource, yshift=1.4cm] {$\sum$};
  	\draw [->] (sigsource.east) -- ++(1,0) |- (sum);
  	\draw [->] (noisesource.east) -- ++(1,0) |- (sum);
  	\node (noisesig) [grblock, right=of sum] {\footnotesize{Archivo \\ ruido\_bb.dat}}
  	edge [<-] (sum);
  	\node (noisedat) [grblock, below=of noisesource] {\footnotesize{Archivo ruido.dat}}
  	edge [<-] (noisesource);
  \end{tikzpicture}
  \vspace{0.3in}
  \caption{Grafo que simula un canal AWGN}
  \label{fig:noisegen}
\end{figure}

El grafo toma como entrada caracter\'istica el valor de la relaci\'on de energ\'ia de bit a ruido
$E_b/N_0$ que se va utilizar en el an\'alisis. Los bloques se implementan de la siguiente manera:

\begin{itemize}
  \item \textbf{Fuente de datos de archivo}: Este bloque se implementa con la clase
  \verb|gr.file_source|. Se utiliza para leer el archivo generado del grafo \ref{fig:siggen} que
  contiene los bits modulados.
  \item \textbf{Fuente de ruido}: Este bloque se implementa usando la clase \\
  \verb|gr.noise_source_c|. El prefijo ``c'' indica que trabaja con datos complejos. La clase acepta
  dos par\'ametros de entrada: el tipo de ruido que se quiere generar y la magnitud. El tipo de
  ruido puede ser una de las siguientes constantes:
  \begin{itemize}
    \item \verb|GR_UNIFORM|
    \item \verb|GR_GAUSSIAN|
    \item \verb|GR_LAPLACIAN|
    \item \verb|GR_IMPULSE| 
  \end{itemize}
  Para simular el canal AWGN se utiliz\'o la constante \verb|GR_GAUSSIAN|.
  \item \textbf{Archivo ruido\_bb.dat y Archivo ruido.dat}: Este bloque usa la clase
  \verb|file_sink_c| para guardar los datos complejos que representan la se\~nal mezclada con el
  ruido. El archivo ruido.dat guarda los datos que representan el puro ruido sin la se\~nal
  modulada.
\end{itemize}

Para determinar la magnitud del ruido el programa primero calcula la potencia promedio del archivo
que contiene la se\~nal modulada sin ruido, la energ\'ia por bit y por \'ultimo la magnitud del
ruido. La potencia se calcula utilizando la siguiente expresi\'on:

\begin{equation}\label{eq:bitpower}
P_{prom}=\frac{1}{N}\sum_{n=1}^{N}{x^2(n)}
\end{equation}

La energ\'ia por bit se calcula a partir de la potencia como se muestra en la siguiente
expresi\'on:

\begin{equation}\label{eq:bitenergy}
E_b=\frac{P_{prom}}{f_b}
\end{equation}
donde $f_b$ es la taza de bits en bits por segundo.

El listado \ref{ex:powenerprogram} muestra dos funciones en Python que calculan estos dos
par\'ametros.

\begin{lstlisting}[float, label=ex:powenerprogram, caption={Funciones en Python para calcular la
potencia y la energia promedio de una se\~nal}]
def get_p_avg_watts(fn, is_complex=False):

    f = open(fn)
    d = f.read()
    num_floats = (len(d)/4)
    d = struct.unpack('f'*num_floats, d)
    if is_complex:
        d = d[::2] #toma la parte real
    d_sq = [x*x for x in d]
    return sum(d_sq)/len(d_sq)
    
def get_eb_joules(fn, bits_per_sec, is_complex=False):

    p_avg = get_p_avg_watts(fn, is_complex=is_complex)
    return p_avg / bits_per_sec

\end{lstlisting}

El c\'odigo que implementa el grafo \ref{fig:noisegen} se muestra en el listado \ref{ex:pynoisegen}.

\begin{lstlisting}[float, label=ex:pynoisegen, caption={C\'odigo que implementa el grafo generador
de ruido}, breaklines=true]

class NoiseGen(gr.top_block):
  def __init__(self, ebn0_dB, bits_per_sec, samp_per_sec):
    gr.top_block.__init__(self, 'NoiseGen')

    ebn0_ratio = dB_to_ratio(ebn0_dB)

    n0_watts_per_Hz = bertool.get_eb_joules(fn = 'limpio_bb.dat', bits_per_sec = bits_per_sec) / ebn0_ratio 
    n_mag = sqrt(0.5 * n0_watts_per_Hz * samp_per_sec)
    n_src = gr.noise_source_c(gr.GR_GAUSSIAN, n_mag)
    sig_src = gr.file_source(gr.sizeof_gr_complex, 'limpio_bb.dat')
    adder = gr.add_cc()

    self.connect(sig_src, adder, gr.file_sink(gr.sizeof_gr_complex, 'ruido_bb.dat'))
    self.connect(n_src, (adder, 1))
    self.connect(n_src, gr.file_sink(gr.sizeof_gr_complex, 'ruido.dat'))
\end{lstlisting}

La tercera etapa realiza la demodulaci\'on de las se\~nales capturadas por los grafos anteriores.
Esta etapa consiste en leer la se\~nal mezclada con ruido, aplicar un filtro pasa bajas pre-selector
para seleccionar la porci\'on del canal que nos interesa, demodular la se\~nal recibida y guardar los
datos generados en un archivo para realizar la comparaci\'on. La estructura del grafo que realiza esta
etapa se muestra en la figura \ref{fig:analizer}.

\begin{figure}[htp]
  \centering
  \vspace{0.3in}
  \begin{tikzpicture}[scale=0.8, transform shape]
  	\node (sigsource) [grblock] {\footnotesize{Archivo con se\~nal y ruido}};
  	\node (fir) [grblock, right=of sigsource] {\footnotesize{Filtro pre-selector}}
  	edge [<-] (sigsource);
  	\node (demod) [grblock, right=of fir] {\footnotesize{Demodulador}}
  	edge [<-] (fir);
  	\node (sink) [grblock, below=of demod] {\footnotesize{bits\_demod\_bb.dat}}
  	edge [<-] (demod);
  \end{tikzpicture}
  \vspace{0.3in}
  \label{fig:analizer}
  \caption{Grafo que realiza la demodulaci\'on de la se\~nal simulada}
\end{figure}

El grafo tiene un par\'ametro obligatorio y es el demodulador que se va utilizar para el an\'alisis.
Los dem\'as son opcionales y establecen los par\'ametros de muestras por s\'imbolos, s\'imbolos
por segundo y los par\'ametros de ancho de banda para el filtro pre-selector. Para este an\'alisis
se utilizaron los valores por default. Los resultados del filtro se muestran en la figura
\ref{fig:predetect}.

\begin{figure}[htp]
  \centering
  \includegraphics[scale=0.5]{figs/predetectfilter}
  \caption{Respuesta del filtro pre-selector en el receptor para la medici\'on del BER.}
  \label{fig:predetect}
\end{figure}

Los bloques del grafo se implementan de la siguiente manera:

\begin{itemize}
  \item \textbf{Archivo con se\~nal y ruido}: La clase \verb|gr.file_source_c| se encarga de leer el
  archivo generado por el grafo \verb|NoiseGen| para enviarlo al filtro pre-selector.
  \item \textbf{Filtro pre-selector}: El filtro pasa bajas se implementa con una combinacion de dos
  clases: \verb|gr.firdes.low_pass| y \verb|gr.fir_filter_ccf|. El primero se utiliza para generar
  los coeficientes del filtro apartir de los parametros de entrada que se le especifiquen. Se
  utilizaron los valores default del ejemplo a excepcion de la ventana. Originalmente utilizaba una
  ventana de tipo Hamming y se cambio a una ventana Kaizer ya que esta tuvo mejor rendimiento y
  ayudo a generar una curva de BER mas suave. Los parametros que acepta la clase son la ganancia, la
  frecuencia de muestreo, la frecuencia de corte, el ancho de la banda de transicion y el tipo de
  ventana. La segunda clase implementa en si el filtro fir apartir de los coeficientes generados por
  la clase \verb|gr.firdes.low_pass|. Sus parametros de entrada son el factor de decimaci\'on y los
  coeficientes del filtro.
  \item \textbf{Demodulador}: Este bloque representa el demodulador que se esta evaluando. En este
  caso fueron ambos DQPSK y DBPSK en sus dos versiones. Las clases que los implementan se encuentran
  dentro del paquete \verb|blks2|. Su uso se muestra en el listado \ref{ex:mainber}. La clase toma
  la se\~nal filtrada e inicia el proceso de demodulaci\'on y detecci\'on. Los resultados son una serie de bits
  desempaquetados, osea, bits individuales.
  \item \textbf{Archivo bits\_demod\_bb.dat}: Este bloque usa la clase \verb|gr.file_sink| para
  escribir a un archivo los bits resultantes del bloque demodulador.
\end{itemize}

El codigo que implementa el tercer grafo se muestra en el listado.

\begin{lstlisting}[float, label=ex:analizer, caption={C\'odigo que implementa el grafo demodulador
para el analisis del BER.}, breaklines=true]
class Analyser(gr.top_block):
    def __init__(self, bb_src_fn, test_demod, samp_per_sym = 8,
                 sym_per_sec = 1e6, pre_detect_filt_bt = 0.9, filt_transition_ratio = 0.1):
        gr.top_block.__init__(self, 'Analyser')

        self.bb_src_fn = bb_src_fn
        self.samp_per_sym = samp_per_sym
        samp_per_sec = samp_per_sym * sym_per_sec

        bb_src = gr.file_source(gr.sizeof_gr_complex, bb_src_fn)

        pre_detect_filt_bw = sym_per_sec * pre_detect_filt_bt
        pre_detect_filt_taps = gr.firdes.low_pass(1.0, samp_per_sec,
                pre_detect_filt_bw, filt_transition_ratio * samp_per_sec,
                gr.firdes.WIN_KAISER, 4.5)
        pre_detect_filt = gr.fir_filter_ccf(1, pre_detect_filt_taps)
        self.connect(bb_src, pre_detect_filt, test_demod)

        self.test_demod_dst_fn = 'bits_demod_bb.dat'
        self.dst = gr.file_sink(gr.sizeof_char, self.test_demod_dst_fn)
        self.connect(test_demod, self.dst)
\end{lstlisting}

El programa principal inicializa los tres grafos y realiza una serie de pruebas para generar la
grafica del BER. La metodolog\'ia que se sigui\'o fue generar varios valores de $E_b/N_0$, desde
12db hasta 2db en intervalos de 0.5db y ejecutar los tres grafos para generar los resultados. Estos
resultados se van guardando en un arreglo y posteriormente se grafican. Los par\'ametros que se
utilizaron para la simulaci\'on fueron los siguientes:

\begin{itemize}
  \item \textbf{Numero de bits}: 20000
  \item \textbf{Muestras por s\'imbolo}: 7
  \item \textbf{S\'imbolos por segundo}: $1e6$
  \item \textbf{Exceso de ancho de banda}: 0.75
\end{itemize}

Los resultados de la primera version de los esquemas DQPSK y DBPSK se muestran en la figura
\ref{fig:bernormal}.

\begin{figure}[htp]
  \centering
  \includegraphics[scale=0.7]{figs/bernormal}
  \caption{Grafica comparativa del BER para la primera versi\'on de los esquemas DQPSK y DBPSK.}
  \label{fig:bernormal}
\end{figure}

Los resultados muestran el rendimiento entre los dos esquemas de modulaci\'on a diferentes niveles
de SNR. Debido a que DQPSK utiliza dos bits por simbolo tiene mas probabilidades de generar errores
a niveles muy altos de ruido. Para poder lograr niveles confiables de transmision se requiere mas
potencia que DBPSK.

La segunda version de estos esquemas intentan mejorar el BER por medio de la implementacion de
bancos de filtros polifasicos. La misma metodologia se siguio para su analisis y los resultados se
muestran el la figura \ref{fig:berpoly}.

\begin{figure}[htp]
  \centering
  \includegraphics[scale=0.7]{figs/berpoly}
  \caption{Grafica comparativa del BER para la segunda versi\'on de los esquemas DQPSK y DBPSK.}
  \label{fig:berpoly}
\end{figure}

Como se puede observar, los filtros polifasicos ofrecen una mejora en el rendimiento de ambos
esquemas. A niveles bajos de SNR DQPSK tiene un poco mas de tolerancia que la version anterior para
generar errores. Con estos se demuestra el rendimiento del codigo fuente actual de \emph{GNURadio}.
El codigo del programa principal se muestra en el listado \ref{ex:mainber}.

\begin{lstlisting}[label=ex:mainber, caption={C\'odigo Python de la rutina principal del
analisis del BER.}, breaklines=true]
if __name__ == "__main__":
    num_bits = 20000
    samp_per_sym = 7
    sym_per_sec = 1e6
    samp_per_sec = sym_per_sec * samp_per_sym
    print "Samples per second: ", samp_per_sec
    xmit_bt = 0.75
    ebn0s_dB = list(numpy.arange(12.0, 1.999, -0.5))
    recv_bt = 0.9
    recv_filt_transition_ratio = 0.1

    sg = SigGen(num_bits = num_bits, xmit_bt = xmit_bt, samp_per_symbol = samp_per_sym)
    sg.run()
    del sg

    dbpsk2_demod_bers = []
    dqpsk2_demod_bers = []
    dbpsk2_delay = None
    dqpsk2_delay = None

    for ebn0_dB in ebn0s_dB:
        print ebn0_dB
        ng = NoiseGen(ebn0_dB = ebn0_dB, bits_per_sec = sym_per_sec, samp_per_sec = samp_per_sec)
        ng.run()
        del ng

        #========================
        # Analisis de DQPSK
        #========================

        test_dqpsk2_demod = blks2.dqpsk_demod(samples_per_symbol = samp_per_sym, excess_bw = xmit_bt)

        a = Analyser(bb_src_fn = 'noisy_qpsk2_bb.dat', test_demod = test_dqpsk2_demod, samp_per_sym = samp_per_sym,
                     sym_per_sec = sym_per_sec, pre_detect_filt_bt = recv_bt,
                     filt_transition_ratio = recv_filt_transition_ratio)

        a.run()
        a.close()
        del a

        if dqpsk2_delay is None:
            dqpsk2_delay = bertool.get_delay('src_bits.dat', 'test_demod_dst_bits.dat')
        ber_stats = bertool.get_ber_stats('src_bits.dat', 'test_demod_dst_bits.dat', dqpsk2_delay)
        dqpsk2_demod_bers.append(ber_stats[0])

        #========================
        # Analisis de DBPSK
        #========================

        test_dbpsk2_demod = blks2.dbpsk_demod(samples_per_symbol = samp_per_sym, excess_bw = xmit_bt)

        a = Analyser(bb_src_fn = 'noisy_bpsk2_bb.dat', test_demod = test_dbpsk2_demod, samp_per_sym = samp_per_sym,
                     sym_per_sec = sym_per_sec, pre_detect_filt_bt = recv_bt,
                     filt_transition_ratio = recv_filt_transition_ratio)

        a.run()
        a.close()
        del a

        if dbpsk2_delay is None:
            dbpsk2_delay = bertool.get_delay('src_bits.dat', 'test_demod_dst_bits.dat')
        ber_stats = bertool.get_ber_stats('src_bits.dat', 'test_demod_dst_bits.dat', dbpsk2_delay)
        dbpsk2_demod_bers.append(ber_stats[0])


    ebn0s_dB.reverse()
    dbpsk2_demod_bers.reverse()
    dqpsk2_demod_bers.reverse()

    fig = p.figure()
    p.title(u'Desempeño de DQPSK y DBPSK')
    ax = fig.add_subplot(111)
    ax.semilogy(ebn0s_dB, dqpsk2_demod_bers, 'b', label = 'DQPSK')
    ax.hold(True)
    ax.semilogy(ebn0s_dB, dbpsk2_demod_bers, 'r', label = 'DBPSK')
    ax.hold(False)

    ax.yaxis.grid(True, which = 'minor')
    ax.xaxis.grid(True)

    p.legend()

    p.ylabel('BER')
    p.xlabel('Ebn0_dB')
    p.show()
\end{lstlisting}
%Espectrograma de la transmision
%==========================================================================
\section{Espectrograma de la transmisi\'on}

%TODO: Explicar el uso de la herramienta USRP FFT y como se puede configurar para observar el
% espectrograma. Explicar que se transmitio una portadora de 30Mhz y que se alcanza a obervar las
% bandas laterales de la transmision. Explicar todos (o casi todos) los controles de la aplicacion.

%Diagrama de ojo
%==========================================================================
\section{Diagrama de ojo}

%TODO: Explicar la interpretacion del diagrama de ojo. Explicar que lo que se observa es el
% parametro beta del filtro RRC y su efecto en la señal observada.

%Constelacion observada
%==========================================================================
\section{Constelaci\'on observada}

%TODO: Incluir ambas imagenes de la constelacion para ilustrar los problemas que se tuvieron.
% Explicar el uso de QT (breve) para el desarrollo de interfazes de usuario y los 4 sinks que
% proporciona GR_QT (freq display, waterfall, time domain, constellation).