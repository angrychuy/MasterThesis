%%%%%%%%%%%%%%%%%%%%%%%%%%%%%%%%%%%%%%%%%%%%%%%%%%%%%%%%%%%%%%%%%%%%%%%%
%% THIS FILE SPECIFIES ALL INFORMATION OF THE AUTHOR AND THE THESIS
%%
%% This program may be distributed and/or modified under the
%% conditions of the LaTeX Project Public License, either version 1.2
%% of this license or (at your option) any later version.
%% The latest version of this license is in
%%   http://www.latex-project.org/lppl.txt
%% and version 1.2 or later is part of all distributions of LaTeX
%% version 1999/12/01 or later.
%%
%%%%%%%%%%%%%%%%%%%%%%%%%%%%%%%%%%%%%%%%%%%%%%%%%%%%%%%%%%%%%%%%%%%%%%%%

%% This is the definition of the work thesis and the packages that the Thesis 
%% will gonna use

\documentclass[12pt,Bold,Justify, letterpaper]{uprmclass}
\usepackage[spanish, mexico]{babel}
\usepackage[final]{graphicx}
\usepackage{amssymb,amsmath,amsthm,mathrsfs,keyval,color,psfrag,multirow,lscape}%,overcite}
\usepackage{calrsfs} % beatiful curly letters
\usepackage{pifont}
\usepackage[bookmarks=true,bookmarksopen=true,breaklinks=true,pdftitle={USRP},plainpages=false,pdfauthor={JesusE},colorlinks=true,hypertexnames=false,citecolor=blue,linkcolor=blue,file
color=blue]{hyperref}
\usepackage{graphics}
\usepackage{rotating}   % Package for rotate tables
%\usepackage{subfigure}  %If you want subfigures
\usepackage{apacite}


%%%%%%%%%%%%%%%%%%%%%%%%%%%%%%%%%%%%%%%%%%%%%%%
%% DEFINE STUDENT AND THESIS SPECIFIC INFO   %%
%%%%%%%%%%%%%%%%%%%%%%%%%%%%%%%%%%%%%%%%%%%%%%%




\SetFullName{Jes\'us Espinoza Hern\'andez}
\SetThesisTypes{Maestr\'ia en Ciencias}
\SetThesisType{Master of Science}
\SetDegreeTypes{Maestr\'ia en Ciencias}
\SetDegreeType{Master of Science}
\SetSpecialty{Ingenier\'ia en Electronica}
\SetDepartment{Faculty of Engineering and Chemistry Science}
\SetGradMes{Agosto} 
\SetGradMonth{August}
\SetGradYear{2010}
\SetDepartamento{Facultad de Ciencias Quimicas e Ingenieria}
\SetChair{M.C. Susana Burnes Rudecino}
\SetTitlesp{Analisis del desempe\~no de la modulaci\'on QPSK en el USRP como
herramienta didactica en la ense\~nanza de telecomunicaciones.}
\SetTitle{Analysis of the performance of the USRP platform using the QPSK
modulation scheme as a tool for teaching Telecommunications}

\palabrasclave{Modulador, Demodulador, Radio reconfigurable por software, USRP, QPSK}
\keywords{Modulator, Demodulator, Software defined radio, USRP, QPSK}

%% Signature page members

% \SetNamea{Primer Miembro del Comite Graduado}				% First Member Graduate Commitee
% \SetDegreea{Ph.D}
% \SetNameb{Segundo Miembro del Comite Graduado}			% Second Member Graduate Commitee
% \SetDegreeb{Ph.D}
% \SetNamec{Presidente del Comite Graduado}						% President Graduate Commitee (Normally Chairman)
% \SetDegreec{Ph.D}
% \SetNamed{Representante de Estudios Graduados}			% Graduate Studies Representative
% \SetDegreed{M.S.}
% \SetNameChairDep{Director del Departamento}					% Chairperson of the Department
% \SetDegreeChairDep{Ph.D}

% defs.tex
%==========================================================================

\theoremstyle{plain}
\newtheorem{definition}[subsection]{Definition}    % This defines the Definition enviroment
																									 % Ver capitulo 5.

\newtheorem{example}{Ejemplo}										 % This is example
																									 % Ver capitulo 5.

\newtheorem{theorem}{Teorema}											 % This is the theorem formulation heading
																									 % Ver capitulo 5.

\newtheorem*{proofa}{Prueba}

\hypersetup{urlcolor=blue}			 % Especifica el azul para los hypervinculos. (pags web)

\newcommand{\fn}[1]{\texttt{#1}}						% Estas dos lineas son utiles para el capitulo 4.
\newcommand{\cn}[1]{\texttt{\char92 #1}}
\newcommand{\gnuradio}{\emph{GNURadio}}


% \makeatletter
% \providecommand*{\toclevel@extrachapter}{0}
% \makeatother

%Parametros para desplegar codigo fuente.
\lstset{language=Python,basicstyle=\footnotesize\sffamily,captionpos=b, showstringspaces=false}
%Traduce la palabra listings a listado para los ejemplos de codigo
\renewcommand{\lstlistingname}{Listado}
\renewcommand{\lstlistlistingname}{\'Indice de listados}

%Tikz library defines
% Define block styles used later 
\tikzstyle{optional}=[draw, fill=yellow!20, text width=5em, 
    text centered, minimum height=2em,drop shadow]
\tikzstyle{generic}=[draw, fill=yellow!40, text width=6em, 
text centered, minimum height=2em,drop shadow]
\tikzstyle{ann} = [above, text width=5em, text centered]
% \tikzstyle{grblock} = [optional, text width=5em, fill=red!20, 
%     minimum height=3em, rounded corners, drop shadow]
\tikzstyle{sc} = [optional, text width=7em, 
    minimum height=5em, rounded corners, drop shadow]
\tikzstyle{grblock} = [draw, text width=7em, fill=red!20, 
    minimum size=5em, rounded corners, drop shadow, text centered]
    
% Used to draw an antenna with tikz
\def\antenna{-- +(0mm,4.0mm) -- +(2.625mm,7.5mm) -- +(-2.625mm,7.5mm) -- +(0mm,4.0mm)}
    
% Define distances for bordering
\def\blockdist{2.3}
\def\edgedist{2.5}   % This file contents all the commands of style defined for the thesis such as theorems,
									 % definitions, etc... (if you have no idea what is it. Don't modify anything)
									 % def.tex is a file used to define special enviroments such as a proof of a theorem.