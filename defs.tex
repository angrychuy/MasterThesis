% defs.tex
%==========================================================================

\theoremstyle{plain}
\newtheorem{definition}[subsection]{Definition}    % This defines the Definition enviroment
																									 % Ver capitulo 5.

\newtheorem{example}{Ejemplo}										 % This is example
																									 % Ver capitulo 5.

\newtheorem{theorem}{Teorema}											 % This is the theorem formulation heading
																									 % Ver capitulo 5.

\newtheorem*{proofa}{Prueba}

\hypersetup{urlcolor=blue}			 % Especifica el azul para los hypervinculos. (pags web)

\newcommand{\fn}[1]{\texttt{#1}}						% Estas dos lineas son utiles para el capitulo 4.
\newcommand{\cn}[1]{\texttt{\char92 #1}}
\newcommand{\gnuradio}{\emph{GNURadio }}


\makeatletter
\providecommand*{\toclevel@extrachapter}{0}
\makeatother

%Parametros para desplegar codigo fuente.
\lstset{language=Python,basicstyle=\footnotesize\sffamily,captionpos=b,showstringspaces=false,frame=single}
%Traduce la palabra listings a listado para los ejemplos de codigo
\renewcommand{\lstlistingname}{Listado}
\renewcommand{\lstlistlistingname}{\'Indice de listados}

%Tikz library defines
% Define block styles used later 
\tikzstyle{optional}=[draw, fill=yellow!20, text width=5em, 
    text centered, minimum height=2em,drop shadow]
\tikzstyle{generic}=[draw, fill=yellow!40, text width=6em, 
text centered, minimum height=2em,drop shadow]
\tikzstyle{ann} = [above, text width=5em, text centered]
% \tikzstyle{grblock} = [optional, text width=5em, fill=red!20, 
%     minimum height=3em, rounded corners, drop shadow]
\tikzstyle{sc} = [optional, text width=7em, 
    minimum height=5em, rounded corners, drop shadow]
\tikzstyle{grblock} = [draw, text width=7em, fill=blue!20, 
    minimum size=5em, rounded corners, drop shadow, text centered]
\tikzstyle{op} = [grblock, circle, text width=1.5em]
\tikzstyle{int}=[draw, minimum size=2em, text centered]
    
% Used to draw an antenna with tikz
\def\antenna{-- +(0mm,4.0mm) -- +(2.625mm,7.5mm) -- +(-2.625mm,7.5mm) -- +(0mm,4.0mm)}

%Used to draw a low pass filter block with tikz. Must be used within a matrix of nodes.
\def\filterLP{\node{};
	\draw[line width=1pt] (-6mm, -5mm) to (-6mm, 5mm)
						  (-6mm, -5mm) to (6mm,-5mm)
						  (-6mm, 3mm) to (-1mm, 3mm) arc (90:25:0.2cm) --
						  +(4mm, -7mm);
						  
}

%Used to draw a sumation block with tikz. Must be used within a matrix of nodes.
\def\sumop{\node[circle, minimum size=1.5em]{};
	\draw (0mm, -2.5mm) to (0mm, 2.5mm)
		  (-2.5mm, 0mm) to (2.5mm, 0mm);
}

%Used to draw a multiplication block with tikz. Must be used within a matrix of nodes.
\def\mulop{\node[circle, minimum size=1.5em]{};
    \draw (-2.5mm, -2.5mm) to (2.5mm, 2.5mm)
          (-2.5mm, 2.5mm) to (2.5mm, -2.5mm);
}

\tikzstyle{block} = [draw, fill=white, 
    minimum size=5em, text centered, text width=13mm]
    
% Define distances for bordering
\def\blockdist{2.3}
\def\edgedist{2.5}

% Define new math operators
\DeclareMathOperator{\sinc}{sinc}